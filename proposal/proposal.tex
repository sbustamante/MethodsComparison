\documentclass[a4,useAMS,usenatbib,usegraphicx,12pt]{article}
%External Packages and personalized macros
\include{latex/macros_proposal}
 
\title{{\textbf{Research Proposal for a Master Thesis in Physics}}\\ 
				Verifying the VPH scheme in Galaxy Formation\\ 
				\color{black}\rule{15cm}{0.5mm}}
\author{Sebastian Bustamante Jaramillo}
\date{}
  
\begin{document}
\maketitle
\begin{center}
\includegraphics[trim = 0mm 3.5cm 0mm 3.0cm, clip, keepaspectratio=true,
width=0.7\textheight]{Presentation1.png}
\tiny{Time evolution of a gas cloud in a supersonic wind using a \VPH\ scheme.
Taken from \citep{Hess10}}
\end{center}
\tableofcontents
 
\newpage 

%============================================================================== 
\section{General Information}
\small
\subsection*{Information of the Student}
\begin{tabular}{L!{\VRule}R}
\bf Name		& Sebastian Bustamante Jaramillo\\
\bf Degree		& B.Sc. in Physics, Universidad de Antioquia (2013)\\
\bf E-mail 1	& macsebas33 \textit{at} gmail.com (personal)\\
\bf E-mail 2	& sebastian.bustamante \textit{at} udea.edu.co (academic)\\
\end{tabular}

\vspace{10pt}

More detailed information of the applicant can be found here \url{http://goo.gl/BPZGzK}

\vspace{15pt}  

\subsection*{Information of the Project}
\begin{tabular}{L!{\VRule}R}
\bf Title		& \bf Verifying the VPH scheme in Galaxy Formation\\
\bf Field		& Cosmology, Astrophysics, Physical Sciences \\
\bf Advisor 1	& Professor Juan Carlos Munoz-Cuartas. Universidad de Antioquia, Colombia.\\
\bf University	& Universidad de Antioquia, Master of Physics program \\
\bf Time Frame	& 2 years \\
\end{tabular}
\normalsize
%==============================================================================

%==============================================================================
\section{Abstract}
%==============================================================================
\newpage


%==============================================================================
\section{Introduction}
%==============================================================================
As we understand more deeply the physical processes involved in astrophysical 
phenomena, it becomes necessary to compute complex interactions of a ever 
increasing number of single components. Some prominent examples include 
the large-scale Universe, galaxy evolution, stellar interior, star formation 
and protoplanetary disk dynamics. A common aspect of these examples is that all 
of them can be regarded basically as a fluid mechanic problem.

\

Although the development of analytical approaches has demonstrated to be a
valuable resource for studying these processes, their increasing complexity 
makes necessary to invoke numerical solutions as a more viable alternative. For
this purpose, two different families of hydrodynamics solvers has been explored
and widely used by the astrophysical community. First, a family of 
moving-mesh-based techniques (e.g. \textit{Smoothed Particle Hydrodynamics} 
\SPH\ \citep{Monaghan92}, \textit{Voronoi Particle Hydrodynamics} \VPH\ 
\citep{Hess10}), and a second family of fixed-mesh-based techniques (e.g.
\textit{Adaptive Mesh Refinement} \texttt{AMR} \citep{Berger89}).

\

Due to the Lagrangian character of moving-mesh methods, techniques like \SPH\ 
are easily implemented on a computer. Furthermore, as the physical system 
evolves, the mass particles naturally move into higher density regions, 
providing a self-adjusting spatial resolution. Nevertheless, \SPH\ has been 
shown to produce spurious suppression of fluid instabilities due to its 
kernel-based density estimator, making it unsuitable to model some of the 
dynamics accurately. On the other hand, fixed-mesh methods like \AMR\ are more 
efficient for capturing shock dynamics. However, due to the conservative nature 
of the hydrodynamical equations, a fixed mesh causes a lack of Galilean 
invariance. Furthermore, the sampling of physical properties over the grid 
introduces spurious vorticity to the fluid, making this technique poor suitable 
for studying turbulent flows.

\

A completely new approach to solve hydrodynamical problems was introduced by 
\citet{Springel10} and implemented into the \AREPO\ code. It combines the 
strengths of \AMR\ and \SPH\ but overcomes many of their weaknesses, hence it 
can be though as a mixed technique. \AREPO\ uses a moving mesh based on a 
Voronoi tessellation defined over a set of particles that represents the fluid. 
The geometry of the mesh resembles very closely that of the point distribution,
retaining the self-adaptivity inherent of \SPH\ and also keeping a grid to 
capture shocks like \AMR\ does. These features make \AREPO\ highly accurate for
simulating a wide range of hydrodynamical problems. Nevertheless, there is a 
price to pay for this accuracy, \AREPO\ demands a huge computing time as 
compared with \SPH\ and even \AMR.

\

A very interesting alternative was introduced by \citet{Hess10}, i.e. the 
\textit{Voronoi Particle Hydrodynamics} \VPH\ technique. This approach consists
of an implementation of \SPH\ with a modified density estimator based on the  
\textit{Voronoi Tessellation Field Estimator} \VTFE. The new estimator
has demonstrated to improve substantially the spurious suppression of fluid 
instabilities as well as retaining the computational efficiency of the original 
formulation.

\

Finally, galaxy evolution and large-scale structure formation are very rich 
astrophysical phenomena where a plethora of hydrodynamical processes can be 
found and studied. In this fashion, cosmological simulations are a quite 
suitable scenario for performing detailed physical and computational 
comparisons of all above-mentioned techniques. It is especially interesting to 
quantify the computational performance of the \VPH\ technique in terms of its 
physical accuracy as compared with the classic approaches and \AREPO.



%==============================================================================
\section{Objectives}
%==============================================================================
\subsection*{General Objective}
Quantifying the computational performance of the \VPH\ technique in terms of 
its physical accuracy for a cosmological setup.


\subsection*{Specific Objectives}
\begin{itemize}
\item Evaluating the physical accuracy provided by \VPH\ for a cosmological 
setup as compared with \AMR, \SPH\ and \AREPO.
\item Exploring and quantifying the differences between \VPH\ and \AMR\ for 
describing shock dynamics in specific hydrodynamical instabilities.
\item Exploring and quantifying the differences of \VPH\ and \SPH\ for 
describing turbulent flows.
\item Measuring the computational performance of \VPH\ as compared with \AREPO.
\end{itemize}


%==============================================================================
\section{Theoretical Framework}
%==============================================================================


%==============================================================================
\section{Methodology}
%==============================================================================
The proposed project is subject to a M.Sc. study and will cover the following 
steps:


\begin{itemize}
\item[\checkmark] \textit{First, a bibliographic review of the original papers 
where were formulated each of the discussed methods should be done. Also a 
review of previous comparison projects.}
\end{itemize}


Before carrying out our enterprise in quantifying the performance of \VPH\ over
cosmological setups, it is necessary to understand deeply the foundations of 
the classic approaches. At this point, a detailed bibliographic review of the 
original papers (for \SPH, \AMR, \VPH\ and \AREPO) should be done. Although no 
previous works have been done in comparing thoroughly the performance of \VPH\
with other approaches over cosmological setups, there are a plenty of comparison
projects for the classic approaches and even \AREPO\ over galaxy simulations and 
commonly used benchmark problems. This literature has to be reviewed as well.

\

\begin{itemize}
\item[\checkmark] \textit{Second, a design of the numerical experiments should 
be done at this point. This includes making cosmological simulations using 
different techniques and if necessary, constructing and simulating specific 
benchmark problems like fluid instabilities.}
\end{itemize}


As this project will be entirely based on numerical results, computing a set of
cosmological simulations as well as some benchmark problems is one of the key 
steps. For this purpose, we will use some packages like \texttt{GADGET} 
\citep{Springel05} for \SPH\ simulations, \texttt{RAMSES} \citet{Teyssier02} 
for \AMR\ and a modified version of \texttt{GADGET} for \VPH.


%==============================================================================
\section{Expected Results}
%==============================================================================
At the end of the stipulated development time for this project, we hope to have
obtained the following results:
\begin{itemize}
\item A toolbox of codes to study the performance of hydro-solvers over 
cosmological setups and over standard benchmark problems in fluid mechanics.
\item A set of cosmological simulations computed by using each of the studied
techniques.
\item A M.Sc. thesis.
\item Submitting a first-author paper to an international journal.
\item Participating with a poster or an oral presentation in an international
event.
\end{itemize}

%==============================================================================
\section{Scientific Impact}
%==============================================================================


%==============================================================================
\section{Schedule}
%==============================================================================


\begin{table}[h]
\begin{flushleft}
\begin{center}
  \begin{tabular}{l  l} \hline\hline
	\centering\textbf{Semester} & \textbf{Goals} \\ \hline
	%First year
	First  
	& \tabitem Identifying a set of existing \texttt{AREPO} simulations suitable 
	for our succeeding \\
	& \ \ \ \ studies. \\
	& \tabitem Applying web finding schemes (T-web and V-web) to the simulations 
	for\\
	& \ \ \ \ quantifying structures in the gaseous cosmic web, i.e. voids, walls, 
	filaments\\
	& \ \ \ \ and clusters.\\
	& \tabitem Evaluating properties of found structures at different redshifts.\\
	\\
	%Second year	
	Second
	& \tabitem Studying by mean of high resolution simulations the impact of the 
	gaseous\\
	& \ \ \ \ cosmic web on specific galaxy evolution processes.\\
	\\	
	
	\hline\hline
  \end{tabular}  
\end{center}
\end{flushleft}
\end{table}


%==============================================================================
\newpage
\bibliographystyle{latex/mn2e}
\renewcommand{\bibname}{8\ \ \ \ Bibliography}
\small
\bibliography{references.bib}
%==============================================================================



\end{document}
