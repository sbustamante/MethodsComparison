\documentclass[a4,useAMS,usenatbib,usegraphicx,12pt]{article}
%External Packages and personalized macros
\include{latex/macros_proposal}
 
\title{{\textbf{Research Proposal for a Master Thesis in Physics}}\\ 
				Verifying the VPH scheme in Galaxy Formation\\ 
				\color{black}\rule{15cm}{0.5mm}}
\author{Sebastian Bustamante Jaramillo}
\date{}
  
\begin{document}
\maketitle
\begin{center}
\includegraphics[trim = 0mm 3.5cm 0mm 3.0cm, clip, keepaspectratio=true,
width=0.7\textheight]{Presentation1.png}
\tiny{Time evolution of a gas cloud in a supersonic wind using a \VPH\ scheme.
Taken from \citep{Hess10}}
\end{center}
\tableofcontents
 
\newpage 

%============================================================================== 
\section{General Information}
\small
\subsection*{Information of the Student}
\begin{tabular}{L!{\VRule}R}
\bf Name		& Sebastian Bustamante Jaramillo\\
\bf Degree		& B.Sc. in Physics, Universidad de Antioquia (2013)\\
\bf E-mail 1	& macsebas33 \textit{at} gmail.com (personal)\\
\bf E-mail 2	& sebastian.bustamante \textit{at} udea.edu.co (academic)\\
\end{tabular}

\vspace{10pt}

More detailed information of the applicant can be found here \url{http://goo.gl/BPZGzK}

\vspace{15pt}  

\subsection*{Information of the Project}
\begin{tabular}{L!{\VRule}R}
\bf Title		& \bf Verifying the VPH scheme in Galaxy Formation\\
\bf Field		& Cosmology, Astrophysics, Physical Sciences \\
\bf Advisor 1	& Professor Juan Carlos Munoz-Cuartas. Universidad de Antioquia, Colombia.\\
\bf University	& Universidad de Antioquia, Master of Physics program \\
\bf Time Frame	& 2 years \\
\end{tabular}
\normalsize
%==============================================================================

%==============================================================================
\section{Abstract}
%==============================================================================
\newpage


%==============================================================================
\section{Introduction}
%==============================================================================
As we understand more deeply the physical processes involved in astrophysical 
phenomena, it becomes necessary to compute complex interactions of a ever 
increasing number of single components. Some prominent examples include 
simulations of the large-scale Universe, galaxy evolution, stellar interior, 
star formation and protoplanetary disk dynamics. A common aspect of these 
examples is that all of them can be regarded basically as a fluid mechanic 
problem.

\

Although the development of analytical approaches has demonstrated to be a
valuable resource for studying these processes, their increasing complexity 
makes necessary to invoke numerical solutions as a more viable alternative. For
this purpose, two different families of hydrodynamics solvers has been explored 
and widely used throughout the literature. First, a family of moving-mesh-based 
techniques (e.g. \textit{Smoothed Particle Hydrodynamics} \SPH\ 
\citep{Monaghan92}, \textit{Voronoi Particle Hydrodynamics} \VPH\ 
\citep{Hess10}), and a second family of fixed-mesh-based techniques (e.g.
\textit{Adaptive Mesh Refinement} \texttt{AMR} \citep{Berger89}).

\

Due to the Lagrangian character of moving-mesh methods, techniques like \SPH\ 
are easily implemented on a computer. Furthermore, as the physical system 
evolves, the mass particles naturally move into higher density regions, 
providing a self-adjusting spatial resolution. However, \SPH\ has been shown 
to produce spurious suppression of fluid instabilities due to its kernel-based
density estimator, making it unsuitable to model some of the dynamics 
accurately. 

\

On the other hand, fixed-mesh methods like the \AMR\ technique are more 
efficient for capturing shock dynamics. However, due to the conservative nature 
of the hydrodynamical equations, the fixed mesh causes a lack of Galilean 
invariance. Furthermore, the sampling of physical properties over the grid 
introduces spurious vorticity to the fluid, making this technique unsuitable 
for studying turbulent flows.



%==============================================================================
\section{Objectives}
%==============================================================================
\subsection*{General Objective}
\begin{itemize}
	\item Evaluating the performance of the \VPH\ method
\end{itemize}


%==============================================================================
\section{Theoretical Framework}
%==============================================================================


%==============================================================================
\section{Methodology}
%==============================================================================


%==============================================================================
\section{Expected Results}
%==============================================================================


%==============================================================================
\section{Scientific Impact}
%==============================================================================


%==============================================================================
\section{Schedule}
%==============================================================================


\begin{table}[h]
\begin{flushleft}
\begin{center}
  \begin{tabular}{l  l} \hline\hline
	\centering\textbf{Semester} & \textbf{Goals} \\ \hline
	%First year
	First  
	& \tabitem Identifying a set of existing \texttt{AREPO} simulations suitable 
	for our succeeding \\
	& \ \ \ \ studies. \\
	& \tabitem Applying web finding schemes (T-web and V-web) to the simulations 
	for\\
	& \ \ \ \ quantifying structures in the gaseous cosmic web, i.e. voids, walls, 
	filaments\\
	& \ \ \ \ and clusters.\\
	& \tabitem Evaluating properties of found structures at different redshifts.\\
	\\
	%Second year	
	Second
	& \tabitem Studying by mean of high resolution simulations the impact of the 
	gaseous\\
	& \ \ \ \ cosmic web on specific galaxy evolution processes.\\
	\\	
	
	\hline\hline
  \end{tabular}  
\end{center}
\end{flushleft}
\end{table}


%==============================================================================
\newpage
\bibliographystyle{latex/mn2e}
\renewcommand{\bibname}{8\ \ \ \ Bibliography}
\small
\bibliography{references.bib}
%==============================================================================



\end{document}
